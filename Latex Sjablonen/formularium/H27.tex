


\section{Magnetisch veld en magnetische krachten}


\begin{tabular}{ p{5cm} p{0.1cm} p{5.6cm}  p{6.5cm} }

Lorentzkracht & & {\begin{align*} \vec{F} &= q\vec{v} \times \vec{B} \\
&= qv_{\perp}B = qvB_{\perp} \\ &= qvB\sin \phi
\end{align}} &  {$\phi = (\widehat{\vec{v},\vec{B}})$ \newline duim = $\vec{F}$ \newline wijsvinger = $\vec{v}$ \newline middelvinger = $\vec{B}$} \\

Magnetische flux & & {\begin{align*} \Phi_{B}  &= \int \vec{B} \cdot d\vec{A} \\ &= \int B_{\perp}dA = \int B \cos \phi dA   \end{align}} & \\

Wet van Gauss & & {\begin{align*} \oint \vec{B}  \cdot d\vec{A} = \vec{0} \end{align}}& De totale magnetische flux door elk gesloten systeem is nul \\

Circulaire beweging in magneetveld & & {\begin{align*} R = \frac{mv}{\abs{q}B} \hspace{1cm} \omega = \frac{v}{R}=\frac{\abs{q}B}{m} \end{align}} & $R$ = straal cirkelbeweging \newline $\omega$ = hoeksnelheid\\

{Snelheid deeltje zonder deflectie} & &  {\begin{align*} v = \frac{E}{B} = \sqrt{\frac{2eV}{m}}\end{align}} & p. 919\\

Lorentzkracht rechte stroomkabel & &  {\begin{align*} \vec{F} = I\vec{l} \times \vec{B} \hspace{1cm} d\vec{F} = Id\vec{l} \times \vec{B} \end{align}} & $\vec{l}$ = vector lengte van het segment \\

Magnetisch dipoolmoment & & {\begin{align*} \mu = IA \end{align}} & \\

Torsie op stroomlus &  & {\begin{align*} \vec{\tau} &= \vec{\mu} \times \vec{B} \\ &= \mu B \sin \phi \\ &=IBA\sin \phi \end{align}} & $A$ = oppervlakte loop \newline $\phi$ = hoek tussen normaal $A$ en $\vec{B}$ \\
{\vspace{-7mm} \begin{flushleft} Potentiële energie magnetische dipool \end{flushleft}} & & {\begin{align*} U = -\vec{\mu} \cdot \vec{B} = -\mu B \cos \phi \end{align}} & $\phi = (\widehat{\vec{\mu},\vec{B}})$ \\

Hall effect & & {\begin{align*} nq = \frac{-J_{x}B_{y}}{E_{x}} \end{align}} & p. 932\\
\end{tabular}